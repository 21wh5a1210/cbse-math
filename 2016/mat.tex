\begin{enumerate}
	\item If $A$ is a square matrix such that $\abs{A} = 5$, write the value of 
$\abs{AA^{\text{T}}}$

	\item $A = \myvec{1 & 2 \\ 3 & -1}$ and $B = \myvec{1 & -4 \\ 3 & -2}$, find $\abs{AB}$.

	\item If $A = \myvec{0 & 3 \\ 2 & -5}$ and $KA = \myvec{0 & 4a \\ -8 & 5b}$ find the values of $k$ and $a$.

	\item Ishan wants to donate a rectangular plot of land for a school in his village. When he was asked to give dimensions of the plot, he told that if its length is decreased by $50m$ and breadth is increased by $50m$, then its area will remain same, but if length is decreased by $10m$ and breadth is decreased by $20m$, then its area will decrease by $5300m^2$. Using matrices, find the dimensions of the plot. Also give reason why he wants to donate the plot for a school.

	\item Using the properties of determinants, prove that:
	\begin{align*}
		\mydet{(b+c)^2 & a^2 & bc \\
		(c+a)^2 & b^2 & ca\\
		(a+b)^2 & c^2 & ab} = (a - b) (b-c) (c-a) (a+b+c) (a^2 + b^2 + c^2)
	\end{align*}
	
	\item Using elementary row operations, find the inverse of the following matrix :
	\begin{align*}
		A = \myvec{2 & -1 & 3\\
		-5 & 3 & 1 \\
		-3 & 2 & 3}
	\end{align*}



	\item If $A = \myvec{\cos \alpha & \sin \alpha\\ -\sin \alpha & \cos \alpha}$, find $\alpha$ satisfying $0<\alpha<\frac{1}{2}$ when $A + A^{\text{T}} = \sqrt{2}I_{2}$, where $A^{\text{T}}$ is transpose of $A$

	\item If $A$ is a $3\times3$ matrix and $\abs{3A} = k \abs{A}$ then write the value of $k$

	\item Using properties of determinants, prove that
	\begin{align*}
		\mydet{(x + y)^2 & zx & zy \\
		zx & (z+y)^2 & xy \\
		zy & xy & (z+x)^2}
		 = 2xyz (x + y + z)^3
	\end{align*}
	
	\item If 
	\begin{align*}
		A = \myvec{1 & 0 & 2\\
		0 & 2 & 1 \\
		2 & 0 & 3}
	\end{align*}
and $A^3-6A^2+7A+kI_3=0$ find $k$.
 \item Use elementary column operation $C_2 \rightarrow C_2 + 2C_1$ in the following matrix equation:
          \begin{align*}
              \myvec{2 & 1 \\2&1} = \myvec{3&1 \\ 2&0} \myvec{1 & 0\\ -1 & 1}
          \end{align*}
    \item Using elementary row operations find the inverse of matrix
          \begin{align*}
              A =\myvec{3 & -3 & 4 \\2&-3&4\\0&-1&1}
          \end{align*}
          and hence solve thr following system of equations
          \begin{align*}
              3x-3y+4z & =21 \\
              2x-3y+4z & =20 \\
              -y+z     & =5.
          \end{align*}
    \item Write the number of all possible matrices of order $2\times 3$ with each entry $1$ or $2$.
    \item Write the number of all possible matrices of order $2\times2$ with each entry $1,2$ or $3$.
    \item A shopkeeper has $3$ varieties of pens $A$, $B$ and $C$. Meenu purchased $1$ pen of each variety for a total of \rupee $21$. Jeevan purchased $4$ pens of $A$ variety, $3$ pens of $B$ variety and $2$ pens of $C$ variety for \rupee $60$. While Shikha purchased $6$ pens of $A$ variety , $2$ pens of $B$ variety and $3$ pens of $C$ variety for \rupee $70$. Using matrix method, find cost of each variety of pen.
    \item If
          \begin{align*}
              A  & =\myvec{1      & -2 & 3 \\-4&2&5} \text{ and}\\
              B  & =\myvec{2      & 3      \\4&5\\2&1} \text{ and}\\
              BA & =\brak{b_{ij}}
          \end{align*}
          find $b_{21} + b_{32}$.
    \item On her birthday Seema decided to donate some money to children of an orphanage home. If there were $8$ children less, every one would have got \rupee $10$ more. However, if there were $16$ children more, every one would have got \rupee $10$ less. Using matrix method, find the number of children and the amount distributed by Seema. What values are reflected by Seema's decision ?
    \item A trust invested some money in two type of bonds. The first bond pays $10$\% interest and second bond pays $12$\% interest. The trust received \rupee $2,800$ as interest. However, if trust had interchanged money in bonds, they would have got \rupee $100$ less as interest. Using matrix method, find the amount invested by the trust. Interst received on this amount will be given to Helpage India as donation. Which value is reflected in this question ?
	\item Solve for x:
          \begin{align*}
              \mydet{a+x & a-x & a-x \\a-x&a+x& a-x\\a-x & a-x & a+x} &=0
          \end{align*}
          using properties of determinants.
    \item If $x \in N$ and
          \begin{align*}
              \mydet{x+3 & -2 \\ -3x & 2x} &= 8
          \end{align*}
          then find the value of $x$.

    \item Using Properties of determinants, show that $\triangle ABC$ is isosceles if :
          \begin{align*}
              \mydet{
              1                 & 1                 & 1                       \\
              1+\cos A          & 1+ \cos B         & 1+ \cos C               \\
              \cos^2 A + \cos A & \cos^2 B + \cos B & \cos^2 C + \cos C} & =0
          \end{align*}
    \item Write the value of
          \mydet{a-b & b-c & c-a \\
              b-c & c-a & a-b\\
              c-a & a-b & b-c
          }.
\item Write the number of all possible matrices of order $2\times2$ with each entry $1, 2$ or $3$.
\item If $ x \in N$ and $\mydet{x+3 & -2 \\ -3x & 2x} = 8$, then find the value of $x$.
\item Use elementary column operation $ C_2 \rightarrow C_2  + 2C_1$ in the following matrix equation:
	\begin{align*} \myvec{2 & 1 \\ 2 & 0} &= \myvec{3 & 1 \\ 2 & 0}\myvec{1 & 0 \\ -1 & 1} \end{align*}
\item Using properties of determinants, show that $\triangle{ABC}$ is isosceles if:
	\begin{align*}\mydet{1 & 1 & 1 \\ 1 + \cos{A} & 1 + \cos{B} & 1 + \cos{C} \\ \cos^2{A} + \cos{A} & \cos^2{B} + \cos{B} & \cos^2{C} + \cos{C} } = 0\end{align*}	
\item A trust invested some money in two types of bonds. The first bond pays $10\%$ interest and second bond pays $12\%$ interest. The trust received \rupee$2800$ as interest. However if trust had interchanged money in bonds, they would have got \rupee$100$ less as interest. Using matrix method, find the amount invested by the trust. Interest received on this amount will be given to Helpage India as donation. Which value is reflected in this question?
\item A shopkeeper has $3$ varieties of pens $A$, $B$ and $C$. Meenu purchased $1$ pen of each variety for a total of \rupee$21$. Jeevan purchased $4$ pens of $A$ variety, $3$ pens of $B$ variety and $2$ pens of $C$ variety for \rupee$60$. While Shikha purchased $6$ pens of $A$ variety, $2$ pens of $B$ variety and $3$ pens of $C$ variety for \rupee$70$. Using matrix method, find cost of each variety of pen.
\end{enumerate}
