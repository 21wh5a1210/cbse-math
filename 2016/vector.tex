\begin{enumerate}
	\item If vectors $\overrightarrow{a}$ and $\overrightarrow{b}$ are such that
 $\abs{\overrightarrow{a}} = \frac{1}{2}$, $\abs{\overrightarrow{b}} = \frac{4}{\sqrt{3}}$
 and $\abs{\overrightarrow{a} \times \overrightarrow{b}} = \frac{1}{\sqrt{3}}$, then find 
 $\abs{\overrightarrow{a}.\overrightarrow{b}}$.

	\item If $\overrightarrow{a}$ and $\overrightarrow{b}$ are unit vectors, then what is the angle between 
$\overrightarrow{a}$ and $\overrightarrow{b}$ for $\overrightarrow{a} - \sqrt{2}\overrightarrow{b}$ to be an unit vector ?
	
	\item Find the distance between the planes 
		\begin{align*}
			\overrightarrow{r}.\myvec{2\hat{i}-3\hat{j}+6\hat{k} } - 4 =0
		\end{align*}
	and 
		\begin{align*}
			\overrightarrow{r}.\myvec{6\hat{i}-9\hat{j} +18\hat{k}} +30 =0
		\end{align*}

	\item Given that vectors $\overrightarrow{a}$, $\overrightarrow{b}$, $\overrightarrow{c}$ form a triangle such that 
$\overrightarrow{a} = \overrightarrow{b}+\overrightarrow{c}$. Find $p$, $q$, $r$, $s$ such that area of triangle is $5\sqrt{6}$ where $\overrightarrow{a} = p\hat{i} +q\hat{j}+r\hat{k}$, 
$\overrightarrow{b} = s\hat{i} +3\hat{j}+4\hat{k}$ and $\overrightarrow{c}=3\hat{i} +\hat{j}-2\hat{k}$.
	
	\item Find the co-ordinates of the point where the line $\overrightarrow{r}=(-\hat{i}-2\hat{j}-3\hat{k})+\lambda(3\hat{i} +4\hat{j}+3\hat{k})$ meets the plane which is perpendicular to the vector $\overrightarrow{n}=\hat{i}+\hat{j} +3\hat{k}$ and at a distance of
$\frac{4}{\sqrt{11}}$ from origin.

	
	\item Write the sum of intercepts cut off by the plane $\overrightarrow{r}.\myvec{2\hat{i}+\hat{j}-\hat{k}} - 5 = 0$ on the three axes.

	\item Find $\lambda$ and $\mu$ if
	\begin{align*}
		\myvec{\hat{i} + 3\hat{j} + 9\hat{k}} \times \myvec{3\hat{i} - \lambda \hat{j} + \mu \hat{k}} = \overrightarrow{0}.
	\end{align*}

	\item If $\overrightarrow{a} = 4\hat{i} - \hat{j} +\hat{k}$ and $\overrightarrow{b} = 2\hat{i} - 2\hat{j} + \hat{k}$, then find a unit vector parallel to the vector $\overrightarrow{a}+\overrightarrow{b}$.

	\item Find the equation of the plane which contains the line of intersection of the planes
	\begin{align*}
		\overrightarrow{r}.\myvec{\hat{i} - 2\hat{j} + 3\hat{k}} - 4 &= 0 \text{  and}\\
		\overrightarrow{r}.\myvec{-2\hat{i} + \hat{j} + \hat{k}} + 5 &= 0
	\end{align*}
and whose intercept on $x$-axis is equal to that of on $y$-axis.
\item Find the coordinates of the foot of perpendicular and perpendicular distance from the point $P\brak{4, 3, 2}$ to the plane
          \begin{align*}
              x+2y+3z=2
          \end{align*}
          Also find the image of $P$ in the plane.
    \item Find the angle between the vectors $\vec{a} + \vec{b}$ and $\vec{a}-\vec{b}$ if
          \begin{align*}
              \vec{a} & =2\hat{i}-\hat{j}+3\hat{k} \quad \text{ and} \\
              \vec{b} & = 3\hat{i} + \hat{j} -2\hat{k}
          \end{align*}
          and hence find a vector perpendicular to both $\vec{a}+\vec{b}$ and $\vec{a}-\vec{b}$.
    \item If  $\abs{\vec{a}} = 4 , \abs{\vec{b}}=3$  and $\vec{a}.\vec{b}=6\sqrt{3}$, then find the value of $\abs{\vec{a}\times \vec{b}}$.
    \item Find the position vector of the point which divides the join of points with position vectors $\vec{a}+3\vec{b}$ and $\vec{a}-\vec{b}$ internally in the ratio $1:3$.
    \item Write the position vector of the point which divides the join of the point s with position vectors $3\vec{a} - 2\vec{b}$ and $2\vec{a} + 3\vec{b}$ in the ratio $2:1$.
    \item Write the number of vectors of unit lenght perpendicular to both the vector
          \begin{align*}
              \vec{a} & = 2 \hat{i} + \hat{j} +2\hat{k} \quad\text{ and} \\
              \vec{b} & = \hat{j}+\hat{k}.
          \end{align*}
    \item Find the vector equationof the plane with intercepts $3,-4$ and $2$ on $x,y$ and  $z$-axis respectively.
    \item Find the coordinates of the point where the line through the points $A\brak{3,4,1}$ and $B\brak{5,1,6}$ crosses the $XZ$ plane. Also find the angle which this line makes with the $XZ$ plane.
    \item The two adjecent sides of a parallelogram are $2\hat{i}-4\hat{j}-5\hat{k}$ and $2\hat{i}+2\hat{j}+3\hat{k}$. Find the two unit vectors parallel to its diagonals. Using the diagonal vectors, find the area of the parallelogram.
    \item Find the position vector of the foot of perpendicular and the perpendicular distance from the point $P$ with position vector $2\hat{i}+3\hat{j}+\hat{k}$ to the plane
          \begin{align*}
              \vec{r}\cdot\brak{2\hat{i}+\hat{j}+3\hat{k}} - 26=0
          \end{align*}
          Also find image of $P$ in the plane.
	\item Write the number of vectors of unit length perpendicular to both the vectors $\overrightarrow{a} = 2\hat{i}+\hat{j} + 2\hat{k}$ and $\overrightarrow{b} = \hat{j} + \hat{k}$
	\item Write the position vector of the point which divides the join of points with position vectors $3\overrightarrow{a} - 2\overrightarrow{b}$ and $2\overrightarrow{a} + 3\overrightarrow{b}$ in the ratio $2:1$.
	\item Find the vector equation of the plane  with intercepts $3, -4$ and $2$ on $x, y$ and $z$ axis respectively.
	\item Find the co-ordinates of the point where the line through the points A $\brak{3,4,1} $ and B $\brak{5,1,6}$ crosses the XZ plane. Also find the angle which this line makes with the XZ plane.
	\item The two adjacent sides of a parallelogram  are $2\hat{i} -4\hat{j} -5\hat{k}$ and $2\hat{i} +2\hat{j} +3\hat{k}$. Find the two unit vectors parallel to its diagnols. Using the diagnol vectors, find the area of the parallelogram.
	\item Find the position vector of the foot of perpendicular and the perpendicular distance from the point $P$ with position vector $2\hat{i} +3\hat{j} + 4\hat{k}$ to the plane $\overrightarrow{r}.\brak{2\hat{i} + 3\hat{j} + 4\hat{k}} - 26 = 0 $. Also find the image of $P$ in the plane.
	\item If $\abs{\overrightarrow{a}} = 4, \abs{\overrightarrow{b}} = 3$ and $ \overrightarrow{a}.\overrightarrow{b} = 6\sqrt{3}$, then find the the value of $\abs{\overrightarrow{a}\times \overrightarrow{b}}$.
	\item Write the coordinates of the point which is the reflection of the point $\brak{\alpha,\beta,\gamma}$ in the $XZ$-plane.
	\item Find the position vector of the point which divides the join of points with position vectors $\overrightarrow{a} + 3\overrightarrow{b}$ and $\overrightarrow{a} - \overrightarrow{b}$ internally in the ratio $1:3$.
	\item Show that the lines $\dfrac{x-1}{3} = \dfrac{y-1}{-1} = \dfrac{z+1}{0}$ and $ \dfrac{x-4}{2} = \dfrac{y}{0} = \dfrac{z+1}{3}$ intersect. Find their point of intersection.
	\item Find the angle between the vectors $ \overrightarrow{a} + \overrightarrow{b} $ and $ \overrightarrow{a} - \overrightarrow{b} $ if $ \overrightarrow{a} = 2\hat{i} - \hat{j} +3\hat{k} $ and $ \overrightarrow{b} = 3\hat{i} + \hat{j} - 2\hat{k}$, and hence find a vector perpendicular to both $ \overrightarrow{a} + \overrightarrow{b} $ and $\overrightarrow{a} - \overrightarrow{b}$.
	\item Find the coordinates of the foot of perpendicular and perpendicular distance from the point $P\brak{4,3,2}$ to the plane $x + 2y + 3z = 2$. Also find the image of $P$ in the plane.
	\item Show that the relation $R$ defined by $\brak{a,b}$R$\brak{c,d} \Rightarrow a + d = b + c$ on the $A\times A$, where $A = \cbrak{1,2,3,\ldots,10}$ is an equivalence relation. Hence write the equivalence class $\sbrak{\brak{3,4}}; a,b,c,d \in A$.
 
\item If $\overrightarrow{a} = 4\hat{i} -\hat{j} + \hat{k}$ and $\overrightarrow{b} = 2\hat{i} -2\hat{j} + \hat{k} $, then find a unit vector parallel to the vector $\overrightarrow{a} + \overrightarrow{b} $.

\item Find $\lambda$ and $\mu$ if\\
$\brak{\hat{i} + 3\hat{j} + 9\hat{k}}\times\brak{3\hat{i} - \lambda\hat{j} + \mu\hat{k}} = \overrightarrow{0}$

\item Write the sum of intercepts cut by the plane $\overrightarrow{r}\cdot\brak{2\hat{i} + \hat{j} - \hat{k}} - 5 = 0$ on the three axes.

\item Find the equations of the plane which contains the line of intersection of the planes
\begin{align*}
	\overrightarrow{r}\cdot\brak{\hat{i} -2\hat{j} + 3\hat{k}}-4&=0\\
\overrightarrow{r}\cdot\brak{-2\hat{i} + \hat{j} + \hat{k}}+5&=0 
\end{align*}
and whose intercept on x-axis is equal to that of y-axis.

\end{enumerate}
