\begin{enumerate}[label=\thesection.\arabic*.,ref=\thesection.\theenumi]
\numberwithin{equation}{enumi}
\numberwithin{figure}{enumi}
\numberwithin{table}{enumi}
	\item Probability of happining of an event is denoted by $p$ and probability of non-happening of the event is denoted by $q$. Relation between $p$ and $q$ is
                        \begin{enumerate}
                                \item $p$+$q$=1
                                \item $p$=1, $q$=1
                                \item $p$=$q$-1
                                \item $p$+$q$+1=0
                        \end{enumerate}
			
        \item A girl calculates that the probability of her winning the first prize in a lottery is 0.08. If 6000 tickets are sold, how many tickets has she bought ?
                        \begin{enumerate}
                                \item  40
                                \item  240
                                \item  480
                                \item  750

                        \end{enumerate}
        \item In a group of 20 people, 5 can't swim. If one person is selected at random, then the probability that he/sh can swim, is
                        \begin{enumerate}
                                \item $ \frac {3} {4} $
                                \item $ \frac {1} {3} $
                                \item 1
                                \item $ \frac {1} {4} $
                        \end{enumerate}
        \item A bag contain 4 red, 3 blue and 2 yellow balls. One ball is drawn at random from the bag. Find the probability that drawn ball is
                \begin{enumerate}
                                        \item red
                                        \item yellow
                \end{enumerate}
        \item A bag contain 100 cards numbered 1 to 100.Acard is drawn at random from the b. What is the probability that the number on the card is a perfect cube ?
                        \begin{enumerate}
                                \item $ \frac {1} {20} $
                                \item $ \frac {3} {50} $
                                \item $ \frac {1} {25} $
                                \item $ \frac {7} {100} $
                        \end{enumerate}
        \item If three coins are tossed simultaneously, what is the probability of getting a most one trail ?
                        \begin{enumerate}
                                \item $ \frac {3} {8} $
                                \item $ \frac {4} {8} $
                                \item $ \frac {5} {8} $
                                \item $ \frac {7} {8} $
                        \end{enumerate}
        \item Two dics are thrown together. The probability of getting the difference of numbers on their upper faces equals to 3 is :
                        \begin{enumerate}
                                \item $ \frac {1} {9} $
                                \item $ \frac {2} {9} $
                                \item $ \frac {1} {6} $
                                \item $ \frac {1} {12} $
                        \end{enumerate}
        \item A card is drawn at random from a well-shuffled pack of 52 cards. The probability that the card drawn is not an ace is :

                        \begin{enumerate}
                                \item $ \frac {1} {13} $
                                \item $ \frac {9} {13} $
                                \item $ \frac {4} {13} $
                                \item $ \frac {12} {13} $
                        \end{enumerate}
        \item \textbf{Assertion (A) : } The probability that a leap year has 53 Students is $ \frac {2} {7} $.\\
                \textbf{Reason (R) : } The probability that a non-leap year has 53 Sundays is $ \frac {5} {7} $.

                          \begin{enumerate}
                                  \item Both Assertion (A) and Reason (R) are true and Reason (R) is the correct explanation of Assertion (A).
                                  \item Both Assertion (A) and Reason (R) are true and Reason (R) is not the correct explanation of Assertion (A).
                                  \item Assertion (A) is true but Reason (R) is false.
                                  \item Assertion (A) is false but Reason (R) is true.
                          \end{enumerate}


\end{enumerate}
