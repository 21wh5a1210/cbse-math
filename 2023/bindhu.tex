%\documentclass[12pt,-letter paper]{article}
%\usepackage{amsmath}
%\usepackage{mathtools}
%\usepackage{gensymb}
%\providecommand{\brak}[1]{\ensuremath{\left(#1\right)}}
%\title{Algebra assignment}
%\author{Prof.G V V sharma}
%\date{\today}

%\begin{document}


\begin{enumerate}

\item If $\alpha$ and $\beta$ are the zeroes of the quadratic polynomial $p(x)=x^2-ax-b$, then the value of $\alpha^2 + \beta^2$ is:


\begin{enumerate}
\item $a^2-2b$
\item $a^2+2b$
\item $b^2-2a$
\item $b^2+2a$
\end{enumerate}

\item The below is the Assertion and Reason based question. Two statements are given, one labelled as Assertion(A) and the other is labelled as Reason(R). Select the correct answer to these questions from the codes (a),(b),(c) and (d) as given below.
\begin{enumerate}
\item Both Assertion(A) and Reason(R) are true and Reason(R) is the correct explanation of the Assertion(A).
\item Both Assertion(A) and Reason(R) are true, but Reason(R) is not the correct explanation of the Assertion(A).
\item Assertion(A) is true, but Reason(R) is false.
\item Assertion(A) is false, but Reason(R) is true.\\ 
\textbf{Assertion(A):} The polynomial $p(x)=x^2+3x+3$ has two real zeroes.\\
\textbf{Reason(R):} A quadratic polynomial can have at most two real zeroes.

\end{enumerate}

\item
\begin{enumerate}
\item If 
\begin{align}
    4\cot^2 45\degree - \sec^2 60\degree + \sin^2 60\degree + p = \frac{3}{4}, 
\end{align}
then find the value of $p$.
\item If 
\begin{align}
    \cos A+ \cos^2A=1,
\end{align}then find the value of 
\begin{align}
\sin^2A+\sin^4A.
\end{align}
\end{enumerate}


\item Prove that:\\
\begin{align}
\brak{\frac{1}{\cos\theta}-\cos\theta}\brak{\frac{1}{\sin\theta}-\sin\theta} = \frac{1}{\tan\theta+\cot\theta}
\end{align}


\item The value of k for which the pair of equations $kx=y+2$ and $6x=2y+3$ has infinitely many solutions,
\begin{enumerate}
\item is $k=3$
\item does not exist
\item is $k=-3$
\item is $k=4$
\end{enumerate}


\item If $2\tan A=3$, then the value of $\frac{4sin A + 3\cos A}{4\sin A - 3\cos A}$ is
\begin{enumerate}
\item $\frac{7}{\sqrt{13}}$
\item $\frac{1}{\sqrt{13}}$
\item $3$
\item does not exist
\end{enumerate}


\item If $\alpha$, $\beta$ are the zeroes of a polynomial $p(x)=x^2+x-1$, then $\frac{1}{\alpha}+\frac{1}{\beta}$ equals to 
\begin{enumerate}
\item $1$
\item $2$
\item $-1$
\item $\frac{-1}{2}$
\end{enumerate}



\end{enumerate}
%\end{document}

