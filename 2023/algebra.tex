\documentclass{article}
\usepackage{siunitx}
\usepackage{setspace}
\usepackage{gensymb}          
\usepackage{xcolor}
\usepackage{caption}
%\usepackage{subcaption}
%\doublespacing               
\singlespacing   
\usepackage[none]{hyphenat}   
\usepackage{amssymb} 
%\usepackage{relsize} 
\usepackage[cmex10]{amsmath}  
\usepackage{mathtools}      
\usepackage{amsmath}   
\usepackage{commath}  
%\usepackage{amsthm}    
%\interdisplaylinepenalty=2500 
%\savesymbol{iint}   
%\usepackage{txfonts}
%\restoresymbol{TXF}{iint}  
%\usepackage{wasysym}    
\usepackage{amsthm}   
\usepackage{mathrsfs}
\usepackage{txfonts}
\let\vec\mathbf{}
%\usepackage{stfloats}
\usepackage{float}
\usepackage{cite}
\usepackage{cases}
\usepackage{subfig}
%\usepackage{xtab}
\usepackage{longtable}
\usepackage{multirow}
%\usepackage{algorithm}
\usepackage{amssymb}
%\usepackage{algpseudocode}
\usepackage{enumitem}
\usepackage{mathtools}
%\usepackage{eenrc}
%\usepackage[framemethod=tikz]{mdframed}
\usepackage{listings}
\usepackage{listings}         
\usepackage[latin1]{inputenc}   
%% \usepackage{color}        
%% \usepackage{lscape}       
\usepackage{titling}                 
%\usepackage{fulbigskip}   
\usepackage{tikz}      
\usepackage{graphicx}
\graphicspath{{/Internal storage/Download/FWC
}}
\usepackage{atbegshi}
%http://ctan.org/pkg/atbegshi
\AtBeginDocument{\AtBeginShipoutNext{\AtBeginShipoutDiscard}}
\newcommand{\mydet}[1]{\ensuremath{\begin{vmatrix}#1\end{vmatrix}}}
\providecommand{\brak}[1]{\ensuremath{\left(#1\right)}}
\providecommand{\norm}[1]{\left\lVert#1\right\rVert}
\newcommand{\solution}{\noindent \textbf{Solution: }}
\newcommand{\myvec}[1]{\ensuremath{\begin{pmatrix}#1\end{pmatrix}}}
\let\vec\mathbf
\begin{document}
\begin{center}
\title{ ALGEBRA}
\date{}
\maketitle  
\end{center}
\begin{enumerate}
    \item $\brak{\sec^2\theta - 1}\brak{\csc^2\theta - 1}$  is equal to:
    \begin{enumerate}
        \item $-1$
        \item  $1$
        \item  $0$
        \item  $2$
        \end{enumerate}
    \item The roots of equation 
    \begin{align}
        x^2 + 3x - 10 = 0
    \end{align}
    are:
    \begin{enumerate}
        \item $\brak{2,-5}$
        \item $\brak{-2,5}$
        \item $\brak{2,5}$
        \item $\brak{-2,-5}$
    \end{enumerate}
    \item If $\alpha$ , $\beta$ are zeroes of the polynomial $x^2-1$,then value of $\brak{\alpha+\beta}$ is:
    \begin{enumerate}
        \item $2$
        \item $1$
        \item $1$
        \item $0$
    \end{enumerate}
    \pagebreak
   \item  If $\alpha$, $\beta$ are the zeroes of the polynomial
   \begin{align}
       p(x)=4x^2 - 3x -7
   \end{align}
   ,then $\brak{\frac{1}{\alpha}+\frac{1}{\beta}}$ is equal to:
   \begin{enumerate}
       \item $\frac{7}{3}$
       \item $\frac{-7}{3}$
       \item $\frac{3}{7}$
       \item $\frac{-3}{7}$
   \end{enumerate}
   \item  Find the sum and product of the roots of the quadratic equation 
   \begin{align}
       2x^2-9x+4=0
   \end{align}
   \item Find the discriminant of the quadratic equation 
   \begin{align}
       4x^2-5=0
   \end{align}
   and hence comment on the nature of roots of the equation.
   \item Evaluate $2\sec^2\theta+3\csc^2\theta-2\sin\theta\cos\theta$ if
   \begin{align}
        \theta=45\si{\degree}
   \end{align}
  
   \item If
   \begin{align}
       \sin\theta-\cos\theta=0
   \end{align}
   ,then find the value of $\sin^4\theta+\cos^4\theta$.
\end{enumerate}
\end{document}
