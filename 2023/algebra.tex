%\documentclass{article}
%\begin{document}
\begin{enumerate}
 \item If one zero of the polynomial \begin{align} p(x)=6x^2+37x-(k-2) \end{align} is reciprocal of the other, then find the value of $k$?
 \item Find the value of $'p'$ for which one root of the quadratic equation \begin{align} px^2-14x+18=0 \end{align}is 6 times the other?
 \item 
 \begin{enumerate}
  \item prove that \begin{align} \frac{\sin A-2 \sin^3A}{2\cos^3A-\cos A}=\tan A\end{align} 
  \item \begin{align} \sec A (1-\sin A)(\sec A+\tan A)=1\end{align} 
  \end{enumerate}
  \item Which of the following  quadratic equations has sum of its roots as 4?
  \begin{enumerate}
      \item $2x^2-4x+8=0$ 
      \item $-x^2+4x+4=0$
      \item $\sqrt{2x^2}-\frac{4}{\sqrt{2}}x+1=0$ 
      \item $4x^2-4x+4=0$
   \end{enumerate}
   \item if one zero of the polynomial \begin{align} 6x^2+37x-(k-2)\end{align}  is reciprocal of the other,then what is the value of $k$?
   \begin{enumerate}
       \item -4
       \item -6
       \item 6
       \item 4
   \end{enumerate}
   \item The zeroes of the polynomial \begin{align}p(x)=x^2+4x+3\end{align}  are given by:
   \begin{enumerate}
       \item 1,3
       \item -1,3
       \item 1,-3
       \item -1,-3
   \end{enumerate}
\item If $\alpha$ and $\beta$ are the zeroes of the quadratic polynomial $p(x)=x^2-ax-b$, then the value of $\alpha^2 + \beta^2$ is:


\begin{enumerate}
\item $a^2-2b$
\item $a^2+2b$
\item $b^2-2a$
\item $b^2+2a$
\end{enumerate}

\item The below is the Assertion and Reason based question. Two statements are given, one labelled as Assertion(A) and the other is labelled as Reason(R). Select the correct answer to these questions from the codes (a),(b),(c) and (d) as given below.
\begin{enumerate}
\item Both Assertion(A) and Reason(R) are true and Reason(R) is the correct explanation of the Assertion(A).
\item Both Assertion(A) and Reason(R) are true, but Reason(R) is not the correct explanation of the Assertion(A).
\item Assertion(A) is true, but Reason(R) is false.
\item Assertion(A) is false, but Reason(R) is true.\\ 
\textbf{Assertion(A):} The polynomial $p(x)=x^2+3x+3$ has two real zeroes.\\
\textbf{Reason(R):} A quadratic polynomial can have at most two real zeroes.

\end{enumerate}

\item
\begin{enumerate}
\item If 
\begin{align}
    4\cot^2 45\degree - \sec^2 60\degree + \sin^2 60\degree + p = \frac{3}{4}, 
\end{align}
then find the value of $p$.
\item If 
\begin{align}
    \cos A+ \cos^2A=1,
\end{align}then find the value of 
\begin{align}
\sin^2A+\sin^4A.
\end{align}
\end{enumerate}


\item Prove that:\\
\begin{align}
\brak{\frac{1}{\cos\theta}-\cos\theta}\brak{\frac{1}{\sin\theta}-\sin\theta} = \frac{1}{\tan\theta+\cot\theta}
\end{align}


\item The value of k for which the pair of equations $kx=y+2$ and $6x=2y+3$ has infinitely many solutions,
\begin{enumerate}
\item is $k=3$
\item does not exist
\item is $k=-3$
\item is $k=4$
\end{enumerate}


\item If $2\tan A=3$, then the value of $\frac{4sin A + 3\cos A}{4\sin A - 3\cos A}$ is
\begin{enumerate}
\item $\frac{7}{\sqrt{13}}$
\item $\frac{1}{\sqrt{13}}$
\item $3$
\item does not exist
\end{enumerate}


\item If $\alpha$, $\beta$ are the zeroes of a polynomial $p(x)=x^2+x-1$, then $\frac{1}{\alpha}+\frac{1}{\beta}$ equals to 
\begin{enumerate}
\item $1$
\item $2$
\item $-1$
\item $\frac{-1}{2}$
\end{enumerate}

    \item $\brak{\sec^2\theta - 1}\brak{\csc^2\theta - 1}$  is equal to:
    \begin{enumerate}
        \item $-1$
        \item  $1$
        \item  $0$
        \item  $2$
        \end{enumerate}
    \item The roots of equation 
    \begin{align}
        x^2 + 3x - 10 = 0
    \end{align}
    are:
    \begin{enumerate}
        \item $\brak{2,-5}$
        \item $\brak{-2,5}$
        \item $\brak{2,5}$
        \item $\brak{-2,-5}$
    \end{enumerate}
    \item If $\alpha$ , $\beta$ are zeroes of the polynomial $x^2-1$,then value of $\brak{\alpha+\beta}$ is:
    \begin{enumerate}
        \item $2$
        \item $1$
        \item $1$
        \item $0$
    \end{enumerate}
    \pagebreak
   \item  If $\alpha$, $\beta$ are the zeroes of the polynomial
   \begin{align}
       p(x)=4x^2 - 3x -7
   \end{align}
   ,then $\brak{\frac{1}{\alpha}+\frac{1}{\beta}}$ is equal to:
   \begin{enumerate}
       \item $\frac{7}{3}$
       \item $\frac{-7}{3}$
       \item $\frac{3}{7}$
       \item $\frac{-3}{7}$
   \end{enumerate}
   \item  Find the sum and product of the roots of the quadratic equation 
   \begin{align}
       2x^2-9x+4=0
   \end{align}
   \item Find the discriminant of the quadratic equation 
   \begin{align}
       4x^2-5=0
   \end{align}
   and hence comment on the nature of roots of the equation.
   \item Evaluate $2\sec^2\theta+3\csc^2\theta-2\sin\theta\cos\theta$ if
   \begin{align}
      \theta=45\degree
   \end{align}
   
   \item If
   \begin{align}
       \sin\theta-\cos\theta=0
   \end{align}
   ,then find the value of $\sin^4\theta+\cos^4\theta$.

\end{enumerate}
%\end{document}
