\begin{enumerate}
\item If the quadratic equation
 \begin{align*}
  px^2 – 2 \sqrt{5}px + 15 = 0
 \end{align*}
has two equal roots,then find the value of $p$.

\item In an AP, if 
\begin{align*}
 S_5 + S_7 &= 167\\  
 S_{10}&= 235
\end{align*}
then find the AP, where $S_n$ denotes the sum of its first $n$ terms.
\item Solve for x :
 \begin{align*}
     \sqrt{3}x^2 -2\sqrt{2}x-2\sqrt{3}= 0 
 \end{align*}
\item Solve the following quadratic equation for $x$:
     \begin{align*}
     4x^{2} + 4bx – (a^{2}–b^{2}) = 0
     \end{align*}
\item The $13^{th}$ term of an AP is four times its $3^{rd}$ term. If its fifth term is $16$, then find the sum of its first ten terms.
\item A truck covers a distance of $150$ km at a certain average speed and then covers another $200$ km at an average speed which is $20$ km per hour more than the first speed. If the truck covers the total distance in $5$ hours, find the first speed of the truck.
\item An arithmetic progression $5, 12, 19, ...$ has $50$ terms. Find its last term. Hence find the sum of its last $15$ terms.


