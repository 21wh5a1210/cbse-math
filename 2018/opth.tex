\begin{enumerate}
\item An open tank with a square base and vertical sides is to be constructed from a metal sheet so as to hold a given quantity of water. Show that the cost of material will be least when depth of the tank is half of its width. If the cost is to be borne by nearby settled lower income families, for whom water will be provided, what kind of value is hidden in this question ?

\item A factory manufactures two types of screws $A$ and $B$, each type requiring the use of two machines, an automatic and a hand-operated. It takes $4$ minutes on the automatic and $6$ minutes on the hand-operated machines to manufacture a packet of screws '$A$' while it takes $6$ minutes on the automatic and $3$ minutes on the hand-operated machine to manufacture a packet of screws '$B$'. Each machine is available for at most $4$ hours on any day. The manufacturer can sell a packet of screws '$A$' at a profit of $70$ paise and screws '$B$' at a profit of \rupee $1$. Assuming that he can sell all the screws he manufactures, how many packets of each type should the factory owner produce in a day in order to maximize his profit ? Formulate the above LPP and solve it graphically and find the maximum profit.

\item A company manufactures two types of novelty souvenirs made of plywood. Souvenirs of type $A$ require $5$ minutes each for cutting and $10$ minutes each for assembling. Souvenirs of type $B$ require $8$ minutes each for cutting and $8$ minutes each for assembling. There are $3$ hours and $20$ minutes available for cutting and $4$ hours available for assembling. The profit is \rupee~$50$ each for type $A$ and \rupee$60$ each for type $B$ souvenirs. How many souvenirs of each type should the company manufacture in order to maximize profit ? Formulate the above LPP and solve it graphically and also find the maximum profit.

\item A manufacturer produces nuts and bolts. It takes $1$ hour of work on machine $A$ and $3$ hours on machine $B$ to produce a package of nuts. It takes $3$ hours on machine $A$ and $1$ hour on machine $B$ to produce a package of bolts. He earns a profit of \rupee$35$ per package of nuts and \rupee$14$ per package of bolts. How many packages of each should be produced each day so as to maximise his profit, if he operates each machine for at most $12$ hours a day ? Convert it into an LPP and solve graphically.
\item A dietitian wishes to mix two types of food in such a way the vitamin contents of the mixtures contain at least $8$ units of vitamin A and $10$ units of vitamin C. It costs \rupee$50$ per Kg to produce food I. Food II contain $1$ unit/kg of vitamin A and $2$ units/kg of Vitamin C and it costs \rupee$70$ per kg to produce food II. Formulate this problem as a LPP to minimise the cost of the mixture that will produce the required diet. Also find the minimum cost.

\item A company manufactures two types of novelty souvenirs made of plywood. Souvenirs of type $A$ require $5$ minutes each for cutting and $10$ minutes each for assembling. Souvenirs of type $B$ require $8$ minutes each for cutting and $8$ minutes each for assembling. There are $3$ hours $20$ minutes available for cutting and $4$ hours for assembling. The profit for type $A$ souvenirs is \rupee$100$ each and for type $B$ souvenirs, profit is \rupee$120$ each. How many souvenirs of each type should the company manufacture in order to maximise the profit ? Formulate the problem as a LPP and then solve it graphically.	

\end{enumerate}
