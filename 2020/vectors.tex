%usepackage{siunitx}         
%\usepackage{setspace}        
%\usepackage{gensymb}         
%\usepackage{xcolor}          
%\usepackage{caption}
%\usepackage{subcaption}
%\doublespacing
%\singlespacing
%\usepackage[none]{hyphenat}  
%\usepackage{amssymb}         
%\usepackage{relsize}         
%\usepackage[cmex10]{amsmath}
%\usepackage{mathtools}
%\usepackage{amsmath}
%\usepackage{amsfonts}
%usepackage{amssymb}        
%\usepackage{commath}
%\usepackage{amsthm}
%\interdisplaylinepenalty=2500
%\savesymbol{iint}
%\usepackage{txfonts}%\restoresymbol{TXF}{iint}
%\usepackage{wasysym}
%\usepackage{amsthm}
%\usepackage{mathrsfs}        
%\usepackage{txfonts}
%\let\vec\mathbf{}
%\usepackage{stfloats}
%\usepackage{float}
%\usepackage{cite}
%\usepackage{cases}
%\usepackage{subfig}          
%\usepackage{xtab}
%\usepackage{longtable}
%\usepackage{multirow}
%\usepackage{algorithm}
%\usepackage{amssymb}
%\usepackage{algpseudocode}
%\usepackage{enumitem}
%\usepackage{mathtools}
%\usepackage{eenrc}
%\usepackage[framemethod=tikz]{mdframed}  \usepackage{listings}                
%\usepackage{listings}
%\usepackage[latin1]{inputenc}
%%\usepackage{color}{
%%\usepackage{lscape}
%\usepackage{textcomp}
%\usepackage{titling}
%\usepackage{hyperref}
%\usepackage{fulbigskip}
%\usepackage{tikz}
%\usepackage{graphicx}
%%\lstset{frame=single, \breaklines=true}}
%\let\vec\mathbf{}
%\usepackage{enumitem}
%\usepackage{amsmath}
%\usepackage{graphicx}        
%\usepackage{tfrupee}
%\usepackage{amsmath}         
%\usepackage{amssymb}
%\usepackage{mwe} % for blindtext and example-image-a in example
%\usepackage{wrapfig}
%\providecommand{\mydet}[1]{\ensuremath{\begin{vmatrix}#1\end{vmatrix}}}
%\providecommand{\myvec}[1]{\ensuremath{\begin{bmatrix}#1\end{bmatrix}}}
%\providecommand{\qfunc}[1]{\ensuremath{Q\left(#1\right)}}
%\providecommand{\sbrak}[1]{\ensuremath{{}\left[#1\right]}}
%\providecommand{\lsbrak}[1]{\ensuremath{{}\left[#1\right]}}
%\providecommand{\rsbrak}[1]{\ensuremath{{}\left[#1\right]}}
%\providecommand{\brak}[1]{\ensuremath{\left(#1\right)}}
%\providecommand{\lbrak}[1]{\ensuremath{\left(#1\right.}}
%\providecommand{\rbrak}[1]{\ensuremath{\left.#1\right)}}
%\providecommand{\cbrak}[1]{\ensuremath{\left\{#1\right\}}}
%\providecommand{\lcbrak}[1]{\ensuremath{\left\{#1\right.}}
%\providecommand{\rcbrak}[1]{\ensuremath{\left.#1\right\}}}
%\title{VECTORS}
%\author{KATTELA SHREYA}
%\date{December 2023}        
%\begin{document}             
%\maketitle
%\section{CLASS 10}
\begin{enumerate}
\item The distance between the points $\brak{m,-n}$ and $\brak{-m, n}$ is
\begin{enumerate}
\item $\sqrt{m^{2} + n^{2}}$
\item $ m+n $
\item $ 2\sqrt{m^{2} + n^{2}}$
\item $\sqrt{2m^{2} + 2n^{2}}$
\end{enumerate}
\item The point on the x-axis which is equidistant from $\brak{-4,0}$ and $\brak{10,0}$ is
\begin{enumerate}             
\item $\brak{7,0}$
\item $\brak{5,0}$              
\item $\brak{0,0}$
\item $\brak{3,0}$
\end{enumerate}
\item The centre of a circle whose end points of a diameter are $\brak{-6,3}$ and $\brak{6,4}$ is
\begin{enumerate}
\item $\brak{8,-1}$
\item $\brak{4,7}$
\item $\brak{0,\frac{7}{2}}$
\item $\brak{4,\frac{7}{2}}$
\end{enumerate}
\item $AOBC$ is a rectangle whose three vertices are $\vec{A}\brak{0,-3}$, $\vec{O}\brak{0,0}$ and $\vec{B}\brak{4,0}$. The length of its diagonal is $\rule{3cm}{0.15mm}$.
\item Find the ratio in which the $y-axis$ divides the line segment joining the points $\brak{6,-4}$ and $\brak{-2, -7}$. Also find the point of intersection.
\item Show that the points $\brak{7, 10}$, $\brak{-2, 5}$ and $\brak{3, 4}$ are vertices of an isosceles right triangle.
\end{enumerate}
\section{CLASS 12}
\begin{enumerate}
\item The area of a triangle formed by vertices $\vec{O}$, $\vec{A}$ and $\vec{B}$, where $\overrightarrow{OA}= \hat{i}+2 \hat{j}+3\hat{k}$ and $\overrightarrow{OB}= -3\hat{i} - 2\hat{j} + \hat{k}$ is
\begin{enumerate}
\item $3\sqrt{5}$ sq. units
\item $5\sqrt{5}$ sq. units
\item $6\sqrt{5}$ sq. units
\item $4$ sq. units
\end{enumerate}
\item The coordinates of the foot of the perpendicular drawn from the point $\brak{2,-3,4}$ on the $y-axis$ is
\begin{enumerate}
\item $\brak{2, 3, 4}$
\item $\brak{-2,-3,-4}$
\item $\brak{0,-3, 0}$
\item $\brak{2, 0,4}$
\end{enumerate}
\item The angle between the vectors $\hat{i} - \hat{j}$ and $\hat{j} - \hat{k}$ is
\begin{enumerate}
\item $\frac{-\pi}{3}$
\item $0$
\item $\frac{\pi}{3}$
\item $\frac{2\pi}{3}$
\end{enumerate}
\item If $\mydet{\overrightarrow{a}}= 4$ and $-3 \leq \lambda \leq 2$, then $\mydet{\lambda \overrightarrow a}$ lies in
\begin{enumerate}
\item $\sbrak{0,12}$
\item $\sbrak{2,3}$
\item $\sbrak{8,12}$
\item $\sbrak{-12,8}$
\end{enumerate}
\item The distance between parallel planes $2x + y - 2z - 6 = 0$ and $4x + 2y - 4z = 0$ is $\rule{3cm}{0.15mm}$ units.
\item If $\vec{P}\brak{1,  0, -3}$ is the foot of the perpendicular from the origin to the plane, then the cartesian equation of the plane is $\underline{\hspace{3cm}}$.
\item Find the coordinates of the point where the line $\frac{x-1}{3} = \frac{y+4}{7} = \frac{z+4}{2}$ cuts the $xy-plane$.
\item Find a vector $\overrightarrow{r}$ equally inclined to the three axes and whose magnitude is $3\sqrt{3}$ units.
\item Find the angle between unit vectors $\overrightarrow{a}$ and $\overrightarrow{b}$ so that $\sqrt{3}\overrightarrow{a}$ - $\overrightarrow{b}$ is also a unit vector.
\item Show that the plane $x - 5y - 2z = 1$ contains the line $\frac{x - 5}{3}$ = y = $2 -z$.
\item Find the equation of the plane passing through the points $\brak{1, 0, -2}$,  $\brak{3, -1, 0}$ and perpendicular to the plane $2x - y + z = 8$. Also find the distance of the plane thus obtained from the origin.
\end{enumerate}
%\end{document}
