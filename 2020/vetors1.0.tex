\documentclass[12pt,-letter paper]{article}
\usepackage{siunitx}         
\usepackage{setspace}        
\usepackage{gensymb}         
\usepackage{xcolor}          
\usepackage{caption}
%\usepackage{subcaption}
\doublespacing
\singlespacing
\usepackage[none]{hyphenat}  
\usepackage{amssymb}         
\usepackage{relsize}         
\usepackage[cmex10]{amsmath} 
\usepackage{mathtools}       
\usepackage{amsmath}
\usepackage{amsfonts}        
\usepackage{amssymb}        
\usepackage{commath}
\usepackage{amsthm}
\interdisplaylinepenalty=2500
%\savesymbol{iint}
\usepackage{txfonts}%\restoresymbol{TXF}{iint}
\usepackage{wasysym}
\usepackage{amsthm}
\usepackage{mathrsfs}        
\usepackage{txfonts}
\let\vec\mathbf{}
\usepackage{stfloats}
\usepackage{float}
\usepackage{cite}
\usepackage{cases}
\usepackage{subfig}          
%\usepackage{xtab}
\usepackage{longtable}
\usepackage{multirow}
%\usepackage{algorithm}
\usepackage{amssymb}
%\usepackage{algpseudocode}
\usepackage{enumitem}
\usepackage{mathtools}
%\usepackage{eenrc}
%\usepackage[framemethod=tikz]{mdframed}  \usepackage{listings}                
%\usepackage{listings}
\usepackage[latin1]{inputenc}
%%\usepackage{color}{
%%\usepackage{lscape}
\usepackage{textcomp}
\usepackage{titling}
\usepackage{hyperref}
%\usepackage{fulbigskip}
\usepackage{tikz}
\usepackage{graphicx}
%%\lstset{frame=single, \breaklines=true}}
\let\vec\mathbf{}
\usepackage{enumitem}
\usepackage{amsmath}
\usepackage{graphicx}        
\usepackage{tfrupee}
\usepackage{amsmath}         
\usepackage{amssymb}
\usepackage{mwe} % for blindtext and example-image-a in example
\usepackage{wrapfig}
\providecommand{\mydet}[1]{\ensuremath{\begin{vmatrix}#1\end{vmatrix}}}
\providecommand{\myvec}[1]{\ensuremath{\begin{bmatrix}#1\end{bmatrix}}}
\providecommand{\qfunc}[1]{\ensuremath{Q\left(#1\right)}}
\providecommand{\sbrak}[1]{\ensuremath{{}\left[#1\right]}}
\providecommand{\lsbrak}[1]{\ensuremath{{}\left[#1\right]}}
\providecommand{\rsbrak}[1]{\ensuremath{{}\left[#1\right]}}
\providecommand{\brak}[1]{\ensuremath{\left(#1\right)}}
\providecommand{\lbrak}[1]{\ensuremath{\left(#1\right.}}
\providecommand{\rbrak}[1]{\ensuremath{\left.#1\right)}}
\providecommand{\cbrak}[1]{\ensuremath{\left\{#1\right\}}}
\providecommand{\lcbrak}[1]{\ensuremath{\left\{#1\right.}}
\providecommand{\rcbrak}[1]{\ensuremath{\left.#1\right\}}}
\title{VECTORS}
\author{KATTELA SHREYA}
\date{December 2023}        
\begin{document}             
\maketitle
\section{CLASS 10}
\begin{enumerate}
\item The distance between the points $\brak{m,-n}$ and $\brak{-m, n}$ is
\begin{enumerate}
\item $\sqrt{m^{2} + n^{2}}$
\item $ m+n $
\item $ 2\sqrt{m^{2} + n^{2}}$
\item $\sqrt{2m^{2} + 2n^{2}}$
\end{enumerate}
\item The point on the x-axis which is equidistant from $\brak{-4,0}$ and $\brak{10,0}$ is
\begin{enumerate}             
\item $\brak{7,0}$
\item $\brak{5,0}$              
\item $\brak{0,0}$
\item $\brak{3,0}$
\end{enumerate}
\item The centre of a circle whose end points of a diameter are $\brak{-6,3}$ and $\brak{6,4}$ is
\begin{enumerate}
\item $\brak{8,-1}$
\item $\brak{4,7}$
\item $\brak{0,\frac{7}{2}}$
\item $\brak{4,\frac{7}{2}}$
\end{enumerate}
\item $AOBC$ is a rectangle whose three vertices are $\vec{A}\brak{0,-3}$, $\vec{O}\brak{0,0}$ and $\vec{B}\brak{4,0}$. The length of its diagonal is $\rule{3cm}{0.15mm}$.
\item Find the ratio in which the $y-axis$ divides the line segment joining the points $\brak{6,-4}$ and $\brak{-2, -7}$. Also find the point of intersection.
\item Show that the points $\brak{7, 10}$, $\brak{-2, 5}$ and $\brak{3, 4}$ are vertices of an isosceles right triangle.
\end{enumerate}
\end{document}
