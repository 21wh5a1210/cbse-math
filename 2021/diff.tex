%\documentclass[12pt,-letter paper]{article}
%\usepackage{siunitx}
%\usepackage{setspace}
%\usepackage{gensymb}
%\usepackage{xcolor}
%\usepackage{caption}
%\usepackage{subcaption}
%\doublespacing
%\singlespacing
%\usepackage[none]{hyphenat}
%\usepackage{amssymb}
%\usepackage{relsize}
%\usepackage[cmex10]{amsmath}
%\usepackage{mathtools}
%\usepackage{amsmath}
%\usepackage{commath}
%\usepackage{amsthm}
%\interdisplaylinepenalty=2500
%\savesymbol{iint}
%\usepackage{txfonts}
%\restoresymbol{TXF}{iint}
%\usepackage{wasysym}
%\usepackage{amsthm}
%\usepackage{mathrsfs}
%\usepackage{txfonts}
%\let\vec\mathbf{}
%\usepackage{stfloats}
%\usepackage{float}
%\usepackage{cite}
%\usepackage{cases}
%\usepackage{subfig}
%\usepackage{xtab}
%\usepackage{longtable}
%\usepackage{multirow}
%\usepackage{algorithm}
%\usepackage{amssymb}
%\usepackage{algpseudocode}
%\usepackage{enumitem}
%\usepackage{mathtools}
%\usepackage{eenrc}
%\usepackage[framemethod=tikz]{mdframed}
%\usepackage{listings}
%\usepackage{listings}
%\usepackage[latin1]{inputenc}
%\usepackage{color}{   
%\usepackage{lscape}
%\usepackage{textcomp}
%\usepackage{titling}
%\usepackage{hyperref}
%\usepackage{fulbigskip}   
%\usepackage{tikz}
%\usepackage{graphicx}
%\lstset{
  %frame=single,
  %breaklines=true
%}
%\let\vec\mathbf{}
%\usepackage{enumitem}
%\usepackage{graphicx}
%\usepackage{siunitx}
%\let\vec\mathbf{}
%\usepackage{enumitem}
%\usepackage{graphicx}
%\usepackage{enumitem}
%\usepackage{tfrupee}
%\usepackage{amsmath}
%\usepackage{amssymb}
%\usepackage{mwe} % for blindtext and example-image-a in example
%\usepackage{wrapfig}
%\graphicspath{{figs/}}
%\providecommand{\mydet}[1]{\ensuremath{\begin{vmatrix}#1\end{vmatrix}}}
%\providecommand{\myvec}[1]{\ensuremath{\begin{bmatrix}#1\end{bmatrix}}}
%\providecommand{\cbrak}[1]{\ensuremath{\left\{#1\right\}}}
%\providecommand{\sbrak}[1]{\ensuremath{{}\left[#1\right]}}
%\providecommand{\brak}[1]{\ensuremath{\left(#1\right)}}

%\begin{document}

%\title{Differentiation}

%\date{\today}

\begin{enumerate}
\item The order and degree of the differential equation of the family of parabolas having vertex at origin and axis along positive x-axis is
\begin{enumerate}
    \item $1,1$
    \item $1,2$
    \item $2,1$
    \item $2,2$
\end{enumerate}
\item If $y = \log x$, then $\frac{d^2y}{dx^2}$ =  \rule{30pt}{1pt}.
\item If $y = e^x + e^{-x}$, then show that $\frac{dy}{dx}$ = $\sqrt{y^2 - 4}$.

\item If $y=x^{\sin x }+\sin^{-1}(\sqrt x)$, the find $\frac{dy}{dx}$.
\item Find the intervals in which the function $f$ defined as $f(x) = \sin(x) + \cos(x)$, $0 \leq x \leq 2\pi$ is strictly increasing or decreasing.
\item Prove that the radius of the right circular cylinder of greatest curved surface area which can be inscribed in a given cone is half of that of the cone. 
\item $\lim\limits_{x \to 0}{\frac{e^{-x} - e^x}{x}}$ is equal to
\begin{enumerate}
    \item $2$
    \item $1$
    \item $-1$
    \item $2$
\end{enumerate}


\item The point at which the normal to the curve $ y = x + \frac{1}{x}, X > 0$ is perpendicular to the line $3x - 4y 7 = 0$ is:
	\begin{enumerate}
	\item $(2, \frac{5}{2})$
	\item $(\pm2, \frac{5}{2})$
      	\item $(-\frac{1}{2}, \frac{5}{2})$
      	\item $(\frac{1}{2}, \frac{5}{2})$
  	\end{enumerate}

\item If $y = log(\cos e^x)$, then $\frac{dx}{dy}$ is: 
   
	\begin{enumerate}
	\item $ \cos e^{x-1} $
	\item $ e^{-x} \cos e^x $
      	\item $ e^x \sin e^x $
      	\item $ -e^x \tan e^x $
  	\end{enumerate}

\item The least value of the function $ f(x) = 2\cos x + x $ in the closed interval $[0, \frac{\pi}{2}]$ is:

  	\begin{enumerate}
      	\item $ 2 $ 
      	\item $ \frac{\pi}{6} + \sqrt 3$
      	\item $ \frac{\pi}{2} $
	\item  The least value does not exist. 
  	\end{enumerate}

\item If $ x = a\sec \theta, y = b\tan \theta,$ then $ \frac{d^2y}{dx^2} $ at $ \theta = \frac{\pi}{2}$ is:
  
  	\begin{enumerate}
    	\item $ \frac{-3\sqrt 3b}{a^2} $
    	\item $ \frac{-2\sqrt 3b}{a} $
    	\item $ \frac{-3\sqrt 3b}{a} $
    	\item $ \frac{-b}{3 \sqrt 3a^2 }$
  	\end{enumerate}

\item The derivative of $ \sin^{-1} (2x \sqrt 1 - x^2) $ w.r.t $ \sin^{-1} x,  -\frac{1}{\sqrt 2 } < x < \frac{1}{\sqrt 2},$ is:
  
  	\begin{enumerate}
    	\item $ 2 $
    	\item $ \frac{\pi}{2} -2 $
    	\item $ \frac{\pi}{2} $
    	\item $ -2 $
  	\end{enumerate}

\item The point(s) on the curve  $ y = x^3 - 11x + 5 $ at which the tangent is $ y = x - 11 $ is/are:
  
  	\begin{enumerate}
    	\item $ (-2, 19)$
    	\item $ ( 2, -9)$
    	\item $ (\pm 2, 19) $
    	\item $ (-2 , 19) and (2, -9) $
  	\end{enumerate}

\item For which value of m is the line  $ y = mx + 1 $ a tangent to the curve $ y^2 = 4x $ ?
  
  	\begin{enumerate}
    	\item $ \frac{1}{2} $
    	\item $ 1 $
    	\item $ 2 $
    	\item $ 3 $
  	\end{enumerate}

\item The maximum value  of $ [x(x - 1) + 1]^\frac{1}{3},  0 \le x \le 1 $ is:
  
  	\begin{enumerate}
    	\item $0$
    	\item $\frac{1}{2}$
    	\item $1$
    	\item $\sqrt3  \frac{1}{3}$
  	\end{enumerate}



\end{enumerate}

%\end{document}
