\begin{enumerate}[label=\thesection.\arabic*.,ref=\thesection.\theenumi]
\numberwithin{equation}{enumi}
\numberwithin{figure}{enumi}
\numberwithin{table}{enumi}
\item $\overrightarrow{a}$   and  $\overrightarrow{ b}$ are two unit vectors such that \begin{align} \abs { 2\overrightarrow{ a}+3\overrightarrow{ b}} = \abs{3\overrightarrow{ a} - 2\overrightarrow{ b}}. \end{align} Find the angle between $\overrightarrow{ a }$ and $\overrightarrow{ b }$.
\item If $\overrightarrow{ a}$  and $\overrightarrow{b}$ are two vectors such that  \begin{align}\overrightarrow{a} = \hat{i} - \hat{j} + \hat{k} \end{align}and  \begin{align}\overrightarrow{b} = 2\hat{i} - \hat{j} - 3\hat{k}\end{align} then find the vector $\overrightarrow{c}$, given that \begin{align}\overrightarrow{a} \times \overrightarrow{c} = \overrightarrow{b}\end{align}  and \begin{align}\overrightarrow{a}.\overrightarrow{c}= 4.\end{align}
\item \begin{align} If \abs{\overrightarrow{ a } \times \overrightarrow { b }}^2 + \abs { \overrightarrow{ a } . \overrightarrow{ b }}^2= 400 \end{align} and  \begin{align}\abs { \overrightarrow{ b}} = 5 \end{align} find the value of  $\abs{\overrightarrow{ a }}$. 
\item If \begin{align}\overrightarrow{a} = \hat{i} + \hat{ j} + \hat{ k} , \overrightarrow{a} . \overrightarrow{b} = 1\end{align}  and \begin{align}\overrightarrow{a} \times \overrightarrow{b} = \hat{j} - \hat{k}\end{align},  then find  $\abs{\overrightarrow{b}}$ 
\item If \begin{align}\abs{\overrightarrow{ a}}= 3, \abs{\overrightarrow{ b}} = 2\sqrt{ 3}\end{align}  and \begin{align}\overrightarrow{ a} . \overrightarrow{ b} = 6,\end{align}then find the value of $\abs{\overrightarrow{ a} \times \overrightarrow{ b}}$.
\item $\abs{\overrightarrow{a}} = 8, \abs{\overrightarrow{ b}} = 3$ and $\overrightarrow{a} . \overrightarrow{b} = 12\sqrt{3}$, then the value of  $\abs{\overrightarrow{a} \times \overrightarrow{b}}$ is
\begin{enumerate}                                      
\item  24                                              
\item  144                                             
\item  2                                              
\item  12                                             
\end{enumerate}
\item If$\space$ \begin{align}\overrightarrow{ a} = 2\hat{i} + \hat{j} + 3\hat{k}, \hat{b} = -\hat{i} + 2\hat{j} + \hat{k}\end{align} and \begin{align}\overrightarrow{c} = 3\hat{i} + \hat{j} + 2\hat{k}\end{align}, then find $\overrightarrow{a} . (\overrightarrow{ b} \times \overrightarrow{c})$. 
\item $\overrightarrow{a}, \overrightarrow{ b },\overrightarrow{ c }$  and  $\overrightarrow{ d }$ are four non-zeros vectors such that  $\overrightarrow{a}\times \overrightarrow{b}= \overrightarrow{c} \times \overrightarrow{d}$  and  \begin{align}\overrightarrow{a} \times \overrightarrow{c} = 4\overrightarrow{b} \times \overrightarrow{d}\end{align}, then show that  $(\overrightarrow{ a}-2\overrightarrow{d} \text{ is parallel to}(2\overrightarrow{b}-\overrightarrow{c})$ where \begin{align}\overrightarrow{a} \neq 2\overrightarrow{d}, \overrightarrow{c} \neq 2\overrightarrow{b}\end{align}
\item If \begin{align}\overrightarrow{a} = \hat{i} + \hat{ j} + \hat{ k} , \overrightarrow{a} . \overrightarrow{b} = 1\end{align}  and \begin{align}\overrightarrow{a} \times \overrightarrow{b} = \hat{j} - \hat{k},\end{align}  then find  $\abs{\overrightarrow{b}}$
\item  If $\overrightarrow{ a}$  and  $\overrightarrow{b}$  are two vectors such that \begin{align}\abs{\overrightarrow{a} + \overrightarrow{b}} = \abs{ \overrightarrow{b}},\end{align}then prove that $(\overrightarrow{a} + 2\overrightarrow{b})$  is perpendicular to $\overrightarrow{ a}$.
\item If $\overrightarrow{ a}$ and $\overrightarrow{ b}$ are unit vectors and $\theta$ is the angle between them , then prove that sin \begin{align}\dfrac{\theta}{ 2} = \dfrac{1}{2}\abs{\overrightarrow{ a} - \overrightarrow{ b}}\end{align}
\item If $\overrightarrow{a}$ and $\overrightarrow{b}$  are two unit vectors such that and $\theta$ is the ang le between them, then prove that                       \begin{align}sin \dfrac{ \theta}{2} = \dfrac{1}{2} \abs{\overrightarrow{a} - \overrightarrow{b}} \end{align} 
\item If \begin{align}\overrightarrow{a} = 2\hat{i} + y\hat{j} + \hat{ k}\end{align} and \begin{align}\overrightarrow{ b} = \hat{i} + 2\hat{j}+ 3\hat{k}\end{align} are two vectors for which the vector $(\overrightarrow{a}+\overrightarrow{b})$ is perpendicular to the vector  $(\overrightarrow{a}-\overrightarrow{b})$ then find all the possible values of y.
\item Write the projection of the vector $(\overrightarrow{b}+\overrightarrow{c})$  on the vector  $\overrightarrow{a}$ ,  where \begin{align}\overrightarrow{ a} = 2\hat{i}-2\hat{j}+\hat{k}, \overrightarrow{b} = \hat{i}+2\hat{j}-2\hat{k}\end{align} and \begin{align}\overrightarrow{c} = 2\hat{i}-\hat{j}+4\hat{k}.\end{align}
\item If \begin{align}\overrightarrow{ a } = 2\hat{i} - \hat{ j } +\hat{ k }, \overrightarrow{ b } = \hat{ i } + \hat{ j} - 2\hat{ k }\end{align} and  \begin{align}\overrightarrow{ c } = \hat{ i } +3\hat{j} - \hat{k}\end{align} and the projection of vector   $\overrightarrow{c} + \lambda \overrightarrow{b}$  on  vector  $\overrightarrow{a}$  is $2\sqrt{6}$, find the value of $\lambda$.
\item If$\space$ $\overrightarrow{ a} = 2\hat{i} + \hat {j} +3\hat{k}, \hat{b} = -\hat{i} + 2\hat{j} + \hat{k }$ and  \begin{align}\overrightarrow{c} = 3\hat{i} + \hat{j} + 2\hat{k}\end{align}, then find $\overrightarrow{a} . (\overrightarrow{ b} \times \overrightarrow{c})$.
\item If $\space$  \begin{align}\overrightarrow { a} = 2\hat{i} - \hat{j} + 2\hat{k}\end{align} and \begin{align}\overrightarrow{ b } = 5\hat{ i } -3\hat{j} -4\hat{k}\end{align}, then find the ratio $\dfrac{ projection  of vector\space \overrightarrow{ a }\space on vector \overrightarrow{ b }}{projection of vector \space \overrightarrow{ b }\space on  vector\space \overrightarrow{ a }}$	
\item Show that the three vectors $2\hat{ i} - \hat{j}  + \hat{k} , \hat{i} - 3\hat{j} - 5\hat{k}$ , and $3\hat{i} - 4\hat{j} - 4\hat{k}$ form the vertices of a right-angled triangle. If $\overrightarrow{ a} = 2\hat{i} + 2\hat{j} + 3\hat{k }, \overrightarrow{ b} = -\hat{i} + 2\hat{j} + \hat{ k }$  and  \begin{align}\overrightarrow{ c} = 3\hat{i} + \hat{ j}\end{align} are such that the vector  $(\overrightarrow{ a} + \lambda \overrightarrow{ b})$ is perpendicular to vector $\overrightarrow{ c}$, then find the value of $\lambda$.	
\item If $\overrightarrow{a} , \overrightarrow{b}$ and  $\overrightarrow{c}$ are the position vectors of the points $\vec{A}(2, 3, -4)$, $\vec{B}(3, -4, -5)$ and $\vec{C}(3, 2,-3)$ and respectively, then $\abs{\overrightarrow{a} + \overrightarrow{b} + \overrightarrow{c}}$ is equal to              
\begin{enumerate}                                     
\item $\sqrt{113}$                                     
\item $\sqrt{185}$                                     
\item $\sqrt{203}$                                     
\item $\sqrt{209}$                                    
\end{enumerate}
\item $\vec{A}$ circle has its center at $(4,4)$. If one end ofa diameter is $(4,0)$, then find the coordinates ofother end.
\item Find the values $\lambda$, for which the distance of point $( 2,1, \lambda)$ from plane \begin{align}3x+5y+4z=11\end{align} is $2\sqrt{2}$ units.                            
\item Find the coordinates of the point where the line through $(3,4,1)$ crosses the ZX-plane
\item Using vectors, find the area of the triangle withvertices $\vec{A}(-1, 0, -2)$, $\vec{B}(0, 2, 1)$ and $\vec{C}(-1, 4,1)$ 
\item Using integration, find the area of triangle region whose vertices are $(2,0)$ , $(4,5)$ and $(1,4)$.
\item The distance between the points $(0,0)$ and $(a-b, a+b)$ is                                             
\begin{enumerate}                                     
\item $2{\sqrt{ab}}$                                  
\item $\sqrt{2a^2 + ab}$                              
\item $ 2\sqrt{a^2 + b^2}$                            
\item $ \sqrt{2a^2 + 2b^2}$                           
\end{enumerate}                                       
\item The value of m which makes the point $(0,0)$ , $( 2m,-4)$and $(3,6)$ collinear, is $\underline{\hspace{1cm}}$
\item  If a line makes $60\degree$  and $45\degree$ angles with the positive directions of X-axis and z-axis respectively, then find the angle that it makes with the positive direction of y-axis. Hence, write the direct6on cosines of the line.
\item The Cartesian equation of a line $AB$ is :         \begin{align}\dfrac{2x-1}{12} = \dfrac{ y+2}{2} = \dfrac{z-3}{3}\end{align}.                        
\item Find the directions cosines of a line parallel to line $AB$.                                             
\item Find the direction cosines of a line whose cartesian equation is given as \begin{align}3x + 1 = 6y - 2 = 1 - z.\end{align}  
\item A vector of magnitude $9$ units in the direction of the vector $-2\hat{i} - \hat{j} + 2\hat{k}$ is \underline{\hspace{1cm}}
\item The two adajacent sides of a parallelogram are represented by $2\hat{i}-4\hat{j}-5\hat{k}$ and $\hat{ i}+2\hat{j}+3\hat{k}$. Find the unit vectors parallel to its diagonals. Using the diagonal vectors, find the area of the parallelogram also.                           
\item The two adjacent sides of a parallelogram are represented by vectors $2\hat{i} - 4\hat{j} + 5\hat{k}$  and  $\hat{ i} - 2\hat{j} - 3\hat{k}$. Find the unit vector parallel to one of its diagonals. Also,find the area of the parallelogram.                               
\item If $\space$ \begin{align}\overrightarrow{ a} = \overrightarrow{i} + 2\overrightarrow{j} + 3\overrightarrow{k}\end{align}   and \begin{align}\overrightarrow{ b} = 2\hat{i} + 4\hat{j} - 5\hat{k}\end{align} represent two adjacent sides of a parallelogram, then find the unit vector parallel to the diagonal of the parallelogram
\item  Find the area of the quadrilateral $ABCD$ whose vertices are $\vec{A}(-4, -3)$ , $\vec{B}(3, -1)$, $\vec{C}(0, 5)$ and $\vec{D}(-4, 2)$                                         
\item If the points $\vec{A}(2,0)$, $\vec{B}(6,1)$, and $\vec{C}(p ,q)$ form a triangle of area 12sq. units (positive only) and \begin{align}2p + q = 10,\end{align}then find the values of p and q.
\end{enumerate}


