\begin{enumerate}
\item Write the number of zeroes in the end of a number whose prime factorization is $2^2 \times 5^3 \times 3^2 \times 17$.

\item Use Euclid's division algorithm to find the HCF of $255$ and $867$.

\item Find the number of terms in the A.P. :
\begin{align*}
    18,15\frac{1}{2},13, ...,-47.
\end{align*}

\item Determine the A.P. whose third term is $16$ and $7^{th}$ term exceeds the $5^{th}$ term by $12$.

\item Find the value of $x$, when in the A.P. given below
\begin{align*}
2 + 6 + 10 + ... + x = 1800.    
\end{align*}

\item Which term of the A.P. $-4, - 1, 2, ... $is$ 101$?

\item In an A.P., the first term is $- 4$, the last term is $29$ and the sum of all its terms is $150$. Find its common difference.

\item Prove that $2 + 3\sqrt{3}$ is an irrational number when it is given that $\sqrt{3}$ is an irrational number.

\item Prove that $2+5\sqrt{3}$ is an irrational number, given that $\sqrt{3}$ is an irrational number.

\item Using Euclid's Algorithm, find the HCF of $2048$ and $960$.

\item If HCF $\brak{336, 54} = 6$, find LCM $\brak{336, 54}$.

\item Write the smallest number which is divisible by both $306$ and $657$.

\item Find the $21^{st}$ term of the A.P. $-4 \frac{1}{2},-3,-1\frac{1}{2},...$

\item Find the common difference of the Arithmetic Progression (A.P.) 
\begin{align*}
\frac{1}{a} , \frac{3-a}{3a},\frac{3-2a}{3a} , . . (a \neq 0)
\end{align*}

\item Which term of the Arithmetic Progression $-7, -12, -17, -22, ... $ will be $-82$ ? Is $-100$ any term of the A.P. ? Give reason for your answer.

\item How many terms of the Arithmetic Progression $45, 39, 33, ... $must be taken so that their sum is $180$ ? Explain the double answer.

\item Find after how many places of decimal the decimal form of the number $\frac {27}{2^3.5^4.3^2}$ will terminate.

\item Express $429$ as a product of its prime factors.

\item If HCF of $65$ and $117$ is expressible in the form $65n - 117$, then find the value of $n$.

\item On a morning walk, three persons step out together and their steps measure $30 cm$, $36 cm$ and $40 cm$ respectively. What is the minimum distance each should walk so that each can cover the same distance in complete steps ?

\item Prove that $\sqrt{3}$ is an irrational number.

\item Find the largest number which on dividing $1251$, $9377$ and $15628$ leaves remainders $1$, $2$ and $3$ respectively.

\item Find the sum of first $10$ multiples of $6$

\item If $m$ times the $m^{th}$ term of an Arithmetic Progression is equal to $n$ times
its $n^{th}$ term and $m \neq n$, show that the $\brak{m + n}^{th}$ term of the A.P is zero

\item The sum of the first three numbers in an Arithmetic Progression is $18$. If the product of the first and the third term is $5$ times the common
difference, find the three numbers.

\item The HCF of two numbers $a$ and $b$ is $5$ and their LCM is $200 $. Find the product $ab$.

\item Find the HCF of $612$ and $1314$ using prime factorisation.

\item Show that any positive odd integer is of the form $6m + 1$ or $6m + 3$ or $6m + 5$, where $m$ is some integer.

\item Prove that ${\sqrt 5}$ is an irrational number.

\item Prove that $\brak{5 - 3{\sqrt 2}}$ is an irrational number, given that ${\sqrt2}$ is irrational number.

\item Find the sum of all the two digit numbers which leave the remainder $2$ when divided by $5$.

\item If in an A.P ., $a=15$,$d=-3$ and $a_n=0$, then find the value of $n$.

\item If ${S_n}$, the sum of the first ${n}$ terms of an A.P. is given by ${S_n = 2n^2 + n}$,then find its $n^{th}$ term. 

\item If the $17^{th}$ term of an A.P. exceeds its $10^{th}$ term by $7$, find the common difference.

\item If the sum of the first $p$ terms of an A.P. is $q$ and the sum of the first $q$ terms is $p$; then show that the sum of the first $\brak{p + q}$ terms is $\cbrak {-\brak {p + q}}$.

\item Write the common difference of the A.P.${\sqrt3} , {\sqrt12} , {\sqrt27} , {\sqrt48}$ , ... 

\item In an A.P., the $n^{th}$ term is ${\frac{1}{m}}$ and the $m^{th}$ term is $\frac{1}{n}$. Find 
\begin{enumerate}
     \item  $\brak{mn}^{th}$term  ,
     \item sum of first $\brak{mn}$ terms.
\end{enumerate}


\end{enumerate}

